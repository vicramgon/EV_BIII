%\section{Introducción}
\justify

Se considera el problema multi-objetivo (sin pérdida de generalidad) consistente en la minimización de $m$ funciones objetivo: $$minimize \, F(\boldsymbol{x})=(f_1(\boldsymbol{x}), \dots, f_m(\boldsymbol{x}))$$ de manera que el espacio de búsqueda $\boldsymbol{x} \in \Omega$ es acotado y corresponde a $\Omega = X_1 \times \cdots \times X_p$, con $X_i$ el conjunto continuo de valores posibles para la componente $x_i$ acotado superior e inferiormente por $x_{Li}$ y $x_{Ui}$ respectivamente.\\

En los problemas de optimización mutiobjetivo, es frecuente que los objetivos sean contrapuestos, esto es, la mejora en uno de los objetivos implica la desmejora de otro, de forma que no existe un único punto que optimice todos los objetivos, si no que se tiene lo que se conoce como frente, en concreto se habla del frente de Pareto o frente Pareto-óptimo.\\

El frente de Pareto corresponde al conjunto de puntos en el espacio de objetivos, tal que ninguno de ellos es dominado por ningún otro de punto de dicho espacio. Dado que un punto $\boldsymbol{u}$ es dominado por otro $\boldsymbol{v}$ si y sólo si $\forall i \in \{1, \dots, m \}: u_i \geq v_i$ y además $\exists j \in \{1, \dots, m\}: u_j > v_j$, esto es $\boldsymbol{u}$ es mejor que $\boldsymbol{v}$ en al menos un objetivo y no es peor en ninguno de los restantes. De forma que el objetivo corresponde, precisamente, a la obtención de este conjunto de puntos. Notaremos la dominancia Pareto como $v \preceq u$.\\

En el estado del arte, existen multitud de aproximaciones a esta categoría de problemas (\textit{MOP}), en este trabajo, se propone abordarla desde la filosofía de los algoritmos evolutivos basados en agregación (\textit{MOEA/D}), que basan su funcionamiento en la descomposición del problema de optimización multi-objetivo en varios subproblemas con un único objetivo (mono-objetivo), de manera que la función objetivo de cada uno de los subproblemas es definida por agregación ponderada de las funciones del problema multi-objetivo. Dentro de este campo existe una amplia variedad de subfilosofías y mejoras al algoritmo original (entre ellas \textit{MOEA/D-DE} \cite{Li2009} o \textit{MOEA/D-DRA} \cite{Qingfu2009}) pero nosotros nos referiremos generalmente a las mismas como \textit{MOEA/D}.\\

Para generar los distintos problemas mono-objetivo existen distintas formulaciones (como \textit{Weighted Sum} o \textit{Boundary Intersections} \cite{Zhang2008}), en este trabajo se seguirá la formulación más corriente en la literatura dada por Tchebychef como: $$ g^{te}\left(\boldsymbol{x}|\boldsymbol{\lambda},\boldsymbol{z^{*}}\right) = \max \limits_{1 \leq i \leq m} \{\lambda_i \, \left|f_{i}(\boldsymbol{x}) - z_{i}^{*}\right|\}$$ donde $\boldsymbol{\lambda}=(\lambda_i, \dots, \lambda_m)$ corresponde a un vector de pesos y $\boldsymbol{z^{*}}=(z^{*}_i, \dots, z^{*}_m)$ corresponde al vector cuyas componentes corresponden a los valores óptimos (mínimos) de las funciones objetivo consideradas de forma aislada, esto es: $$ z_i^{*} = \min \{ f_{i}(\boldsymbol{x}) \, | \,  \boldsymbol{x} \in \Omega \}$$

De forma que, para cada punto $\boldsymbol{x}^{*}$ del frente de Pareto existe un vector de pesos $(\boldsymbol{\lambda})$ tal que  $\boldsymbol{x}^{*}$ es solución óptima del subproblema asociado a $(\boldsymbol{\lambda})$, de manera que recorriendo completamente los posibles $\boldsymbol{\lambda}$ se obtendrían todos los puntos del frente Pareto-óptimo. Computacionalmente, nos quedaremos con una muestra de todos los posibles vectores de pesos $\boldsymbol{\lambda}$, de forma que tratemos de obtener una aproximación al frente de Pareto completo.\\


La filosofía \textit{MOEA/D} se enmarca dentro de los conocidos como \textit{Algoritmos Evolutivos} y como tal uno de los pasos fundamentales del funcionamiento del mismo se basa en la mutación y recombinación de los individuos de la población. En este aspecto la literatura está plagada de distintos operadores, dentro de los cuales los basados en \textit{Evolución Diferencial} son claramente los más numerosos. En este trabajo compararemos tres operadores (\textit{EOP1}, \textit{EOP2}, \textit{EOP3}), sobre un mismo marco de desarrollo, mostrando tanto la metodología seguida como la implementación realizada y los resultados obtenidos, evaluando, mediante el uso de distintas métricas características en la literatura (\textit{Hypervolume}, \textit{Spacing}, \textit{Cober Set}), la eficacia del algoritmo presentado (y los distintos operadores) en relación a otros presentados en este mismo ámbito (\textit{NSGA-II}).\\

%%%%%%%%%%%%%%%%%%%%%%%%%%%%%%%%%%%%%%%%%%%%%%%%%%%%%%%%%%%%

